\documentclass[aspectratio=169,11pt]{beamer}
\usetheme{Warsaw}
\usepackage[utf8]{inputenc}
\usepackage[polish]{babel}
\usepackage{amsmath}
\usepackage{amsfonts}
\usepackage{amssymb}
\usepackage{graphicx}
\usepackage{subfig}
\usepackage{multicol}
\usepackage{mathtools}
\usepackage{booktabs}
\usepackage{pdfpages}
\author{Marcin Skrzypkowski}
\title{Autonomiczna nawigacja manipulatora mobilnego}

%\setbeamercovered{transparent} 
%\setbeamertemplate{navigation symbols}{} 
%\logo{} 
%\institute{} 
%\date{} 
%\subject{} 

\usepackage[
    backend=bibtex,
    style=ieee
]{biblatex}

\addbibresource{sources.bib}

%set page counters
\addtobeamertemplate{navigation symbols}{}{%
    \usebeamerfont{footline}%
    \usebeamercolor[fg]{footline}%
    \hspace{1cm}%
    \insertframenumber/\inserttotalframenumber
}

%define new subsection for display in frames
\setbeamercovered{transparent}

\makeatletter
\let\oldheadcommand\headcommand
\newcommand{\stopnavigation}{\addtocontents{nav}{\string\let\string\headcommand\string\@gobble}}
\newcommand{\resumenavigation}{\addtocontents{nav}{\string\let\string\headcommand\string\oldheadcommand}}


\newcounter{prevsection}
\newcounter{nextsection}

\newcommand\prevsection{}
\newcommand\nextsection{}

\makeatletter
\long\def\beamer@section[#1]#2{%
  \beamer@savemode%
    \mode<all>%
  \ifbeamer@inlecture%
   \refstepcounter{section}%
    \beamer@ifempty{#2}%
    {\long\def\secname{#1}\long\def\lastsection{#1}}%
    {\global\advance\beamer@tocsectionnumber by 1\relax%
      \long\def\secname{#2}%
      \long\def\lastsection{#1}%
      \addtocontents{toc}{\protect\beamer@sectionintoc{\the\c@section}{#2}{\the\c@page}{\the\c@part}%
        {\the\beamer@tocsectionnumber}}}%
    {\let\\=\relax\xdef\sectionlink{{Navigation\the\c@page}{\noexpand\secname}}}%
    \beamer@tempcount=\c@page\advance\beamer@tempcount by -1%
    \beamer@ifempty{#1}{}{%
      \addtocontents{nav}{\protect\headcommand{\protect\sectionentry{\the\c@section}{#1}{\the\c@page}{\secname}{\the\c@part}}}%
      \addtocontents{nav}{\protect\headcommand{\protect\beamer@sectionpages{\the\beamer@sectionstartpage}{\the\beamer@tempcount}}}%
      \addtocontents{nav}{\protect\headcommand{\protect\beamer@subsectionpages{\the\beamer@subsectionstartpage}{\the\beamer@tempcount}}}%
    }%
    \beamer@sectionstartpage=\c@page%
    \beamer@subsectionstartpage=\c@page%
    \def\insertsection{\expandafter\hyperlink\sectionlink}%
    \def\insertsubsection{}%
    \def\insertsubsubsection{}%
    \def\insertsectionhead{\hyperlink{Navigation\the\c@page}{#1}}%
    \def\insertsubsectionhead{}%
    \def\insertsubsubsectionhead{}%
    \def\lastsubsection{}%
    \Hy@writebookmark{\the\c@section}{\secname}{Outline\the\c@part.\the\c@section}{2}{toc}%
    \hyper@anchorstart{Outline\the\c@part.\the\c@section}\hyper@anchorend%
    \beamer@ifempty{#2}{\beamer@atbeginsections}{\beamer@atbeginsection}%
  \fi%
  \beamer@resumemode
    \setcounter{prevsection}{\thesection}%
    \setcounter{nextsection}{\thesection}%
    \addtocounter{prevsection}{-1}%
    \gdef\prevsection{\csname section\romannumeral\theprevsection \endcsname}%
     \addtocounter{nextsection}{1}%
    \renewcommand\nextsection{\csname section\romannumeral\thenextsection \endcsname}%
}%

\setbeamertemplate{headline}
{%
  \leavevmode%
  \@tempdimb=1.2ex%
  \ifnum\beamer@subsectionmax<\beamer@sectionmax%
    \multiply\@tempdimb by\beamer@sectionmax%
  \else%
    \multiply\@tempdimb by\beamer@subsectionmax%
  \fi%
  \ifdim\@tempdimb>0pt%
    \advance\@tempdimb by 1.125ex%
    \begin{beamercolorbox}[wd=.5\paperwidth,ht=\@tempdimb,right,rightskip=1em]{section in head/foot}%
      \vbox to \@tempdimb{%
      \ifnum\thesection=1 \else%
        \vfill{\color{fg!40!bg}\prevsection}%
      \fi%
        \vfill\insertsectionhead%
      \ifnum\thesection=\beamer@sectionmax \else%
        \vfill{\color{fg!40!bg}\nextsection}%
     \fi\vfill%
    }%
    \end{beamercolorbox}%
    \begin{beamercolorbox}[wd=.5\paperwidth,ht=\@tempdimb]{subsection in head/foot}%
      \vbox to\@tempdimb{\vfil\insertsubsectionnavigation{.5\paperwidth}\vfil}%
    \end{beamercolorbox}%
  \fi%
}%
\makeatother

% Here you put the names that will go in the navigation bar
\newcommand\sectioni{Cel pracy}
\newcommand\sectionii{Robot Velma}
\newcommand\sectioniii{Mapa}
\newcommand\sectioniv{Mapa kosztów}
\newcommand\sectionv{Lokalizacja}
\newcommand\sectionvi{Slam}
\newcommand\sectionvii{Planowanie ścieżki}
\newcommand\sectionviii{Navigation Stack}
\newcommand\sectionix{Sformułowanie zadań}
\newcommand\sectionx{Dotychczasowe wyniki}
\newcommand\sectionxi{Zadania do wykonania}
\newcommand\sectionxii{Zakończenie}
\newcommand\sectionxiii{Źródła}


\begin{document}
{
\setbeamertemplate{headline}{}
\begin{frame}
\titlepage
	\begin{center}
		promotor: dr hab. inż. Wojciech Szynkiewicz \\
		\small{Instytut Automatyki i Informatyki Stosowanej}
	\end{center}
\end{frame}
}

{
\setbeamertemplate{headline}{}
%\addtobeamertemplate{frametitle}{\vspace*{-1\baselineskip}}{}
\addtobeamertemplate{frametitle}{\vspace*{-2\baselineskip}}{}
\begin{frame}
\frametitle{Wstęp}
	\begin{figure}[!tbp]
		\centering
  	\subfloat{\includegraphics[width=0.4\textwidth]{img/velma_gazebo.png}}
  	\hfill
  	\subfloat{\includegraphics[width=0.4\textwidth]{img/velma_rviz.png}}
	\end{figure}
\end{frame}
}

{
\setbeamertemplate{headline}{}
\addtobeamertemplate{frametitle}{\vspace*{-2\baselineskip}}{}
\begin{frame}
\frametitle{Plan prezentacji}
	\begin{multicols}{2}
		\tableofcontents
	\end{multicols}
\end{frame}
}

\section{Cel pracy i motywacja}
\begin{frame}
\frametitle{Cel pracy}
	Celem prezentowanej tu pracy inżynierskiej jest opracowanie i implementacja systemu autonomicznej nawigacji dwuręcznego manipulatora mobilnego Velma.
	Do zadań systemu należy budowa mapy otoczenia, lokalizacja w stworzonej mapie środowiska, jednoczesna budowa mapy i lokalizacja, oraz planowanie i bezpieczne wykonanie ścieżki do zadanego punktu w otoczeniu.
	
\end{frame}



\begin{frame}
\frametitle{Motywacja pracy}
Motywacją opisywanej pracy jest rozwój platformy naukowej jaką jest robot Velma, połączenie możliwości skomplikowanego manipulatora z wszechstronną bazą mobilną.
\end{frame}

{
\setbeamertemplate{headline}{}
\addtobeamertemplate{frametitle}{\vspace*{-4\baselineskip}}{}
\begin{frame}
\frametitle{NAWIGACJA}
	\begin{center}
		\LARGE{\textbf{NAWIGACJA}}
	\end{center}
\end{frame}
}

\section{Robot Velma}

\begin{frame}{Symulacja robota}
%	zdjęcie Velmy w środowisku
	\begin{itemize}
		\item Robot Velma - dwa manipulatory LWR Kuki, każdy o siedmiu stopniach swobody
		\item Baza mobilna - wielokierunkowa, wykorzystująca koła szwedzkie
		\item Symulacja na potrzeby prac laboratoryjnych, w początkowych fazach projektów prace na samym robocie są zbyt niebezpieczne
	\end{itemize}
\end{frame}


\begin{frame}{Symulacja części mobilnej}
	\begin{columns}
		\begin{column}{0.5\textwidth}
			\begin{center}
				MODUŁ SYMULACYJNY
			\end{center}
			\begin{itemize}
				\item kształt wizualny obiektu
				\item kolizje obiektu
				\item dynamika obiektu
			\end{itemize}
			
		\end{column}
		\begin{column}{0.5\textwidth}  %%<--- here
			\begin{center}
				MODUŁ STERUJĄCY
			\end{center}
			\begin{itemize}
				\item wykorzystuje informacje z pierwszego modułu
				\item przyjmuje jako wejście prędkość zadaną bazy
				\item oblicza siły działające na bazę
			\end{itemize}
		\end{column}
	\end{columns}
\end{frame}

\begin{frame}{Oprogramowanie części mobilnej}
	\begin{columns}
		\begin{column}{0.5\textwidth}
			\begin{center}
				ROS
			\end{center}
			\begin{itemize}
				\item popularny i rozwijany system programowania robotów
				\item umożliwia tworzenie niezależnych procesów w systemie operacyjnym komputera
				\item komunikacja między procesami z pomocą jednokierunkowych tematów i dwukierunkowych serwisów
			\end{itemize}
			
		\end{column}
		\begin{column}{0.5\textwidth}  %%<--- here
			\begin{center}
				GAZEBO
			\end{center}
			\begin{itemize}
				\item symulowanie obiektów w trzech wymiarach
				\item proste pisanie własnych pluginów w C++
				\item początkowo część ROSa, teraz niezależna platforma
			\end{itemize}
		\end{column}
	\end{columns}
\end{frame}

\section{Mapa}


\begin{frame}{Do czego potrzebna jest mapa}
	Istnieje implementacja zadania nawigacji bez wykorzystania mapy, w tym wypadku jest to tak zwana nawigacja zliczeniowa. 
	Jej wykorzystanie powoduje jednak generowanie błędu pozycji, którego nie da się zniwelować,	a który rośnie z czasem.
	W celu minimalizacji tego rodzaju błędów wykorzystuje się cechy charakterystyczne środowiska \cite{robotics_vision_dead_reckon}, a więc mapę.
	To podejście do nawigacji jest wykorzystywane w opisywanej pracy inżynierskiej.
\end{frame}

\begin{frame}{Jak działa algorytm budowy mapy}
	\begin{itemize}
		\item początkowe założenie - znana pierwotna pozycja robota
		\item uzyskanie punktów otoczenia względem robota
		\item możliwość wykorzystania wielu czujników - systemu LiDAR, kamery Kinect, sonarów etc.
		\item możliwa identyfikacja poszczególnych punktów na mapie w celu późniejszego szybkiego ich rozpoznania lub przechowywanie jedynie współrzędnych zajętych obszarów
	\end{itemize}
\end{frame}

\begin{frame}{Na co pozwalają implementacje budowy mapy}
	\begin{columns}
		\begin{column}{0.5\textwidth}
			\begin{figure}
				\centering
				\includegraphics[height=0.6\textheight]{img/mapa_2d.png}
				\caption{dwuwymiarowa mapa zbudowana przy pomocy czujników LiDAR}
			\end{figure}
		\end{column}
		\begin{column}{0.5\textwidth}  %%<--- here
			\begin{figure}
				\centering
				\includegraphics[height=0.6\textheight]{img/octomapa.png}
				\caption{mapa trójwymiarowa zbudowana za pomocą kamery Kinect}
			\end{figure}
		\end{column}
	\end{columns}
\end{frame}

\begin{frame}{Własności niektórych implementacji}
	\begin{columns}
		\begin{column}{0.5\textwidth}
			\begin{center}
				czujnik LiDAR
			\end{center}
			\begin{itemize}
				\item szybkie tworzenie obrazów dużych płaskich przestrzeni
				\item dane pobrane tylko na jednej wysokości, zwykle blisko podłoża (mapa 2D)
			\end{itemize}
		\end{column}
		\begin{column}{0.5\textwidth}  %%<--- here
			\begin{center}
				kamera Kinect
			\end{center}
			\begin{itemize}
				\item mniejszy kąt widzenia czujnika i wymagane większe zasoby - potrzeba więcej czasu, pamięci i cykli procesora do budowy takiej mapy				
				\item budowa obrazu trójwymiarowej przestrzeni
			\end{itemize}
		\end{column}
	\end{columns}
\end{frame}


\section{Mapa kosztów}

\begin{frame}{Czym jest mapa kosztów}
	\begin{itemize}
		\item mapa - informacje które punkty w przestrzeni są zajęte, ewentualnie informacje o ich identyfikacji
		\item mapa kosztów - każdy punkt w przestrzeni ma przypisaną pewną wartość (binarnie lub z większym przedziałem), wykorzystywana w algorytmach wyszukiwania ścieżki
		\item przez dodawanie wartości pośrednich można sprawić by algorytm unikał nieporządanych miejsc jak pobliże ścian czy śliska nawierzchnia
	\end{itemize}
\end{frame}

\begin{frame}{Mapa i mapa kosztów}
\begin{columns}
		\begin{column}{0.5\textwidth}
			\begin{center}
				MAPA
			\end{center}
%TODO			wstaw obrazek mapy
		\end{column}
		\begin{column}{0.5\textwidth}  %%<--- here
			\begin{center}
				MAPA KOSZTÓW
			\end{center}
%TODO			wstaw obrazek mapy kosztów
		\end{column}
	\end{columns}
\end{frame}

\begin{frame}
{Wielowarstwowe mapy kosztów}
\begin{columns}
		\begin{column}{0.5\textwidth}
			\begin{itemize}
				\item pozwalają na zobrazowanie kosztu poruszania w większej ilości kontekstów (ruch prawostronny, omijanie przestrzeni publicznej)
				\item możliwość wykrywania przeszkód na kilku poziomach (przykładowo wykrycie blatu stołu)
				\item możliwość wykorzystania chmury punktów z Kinecta do stworzenia mapy kosztów przez zrzutowanie wszystkich punktów na płaszczyznę
			\end{itemize}
		\end{column}
		\begin{column}{0.5\textwidth}  %%<--- here
%TODO			wstaw obrazek mapy wielowarstwowej
		\end{column}
	\end{columns}
\end{frame}



\section{Lokalizacja}

\begin{frame}
{Operator przekształcenia jednorodnego}
Macierz przedstawiająca operację przejścia z układu współrzędnych $0$ do $1$.
Dzięki temu można przekształcić pozycję obiektu znajdującego się w układzie $1$ i przedstawić jako pozycję względem układu $0$.

%TODO wstawić r-nie anro-kin no 34
\end{frame}

\begin{frame}
{Drzewo transformacji}
	\begin{itemize}
		\item 4 parametry Denavita-Hartenberga opisujące przejście między dwoma układami współrzędnych (obrót i przesunięcie względem odpowiednich osi $x$ oraz $z$)
		\item macierze wygenerowane z tych parametrów można mnożyć sekwencyjnie, dzięki czemu relatywnie proste jest przedstawianie układów szeregowych z dużą ilością stawów - drzewo powstaje przez przejście z podstawowego układu (przykładowo świata) przez kolejne układy do docelowego
		\item drzewo transformacji może mieć tylko jeden korzeń, każdy węzeł (układ współrzędnych) tylko jednego rodzica, za to wiele dzieci
	\end{itemize}
	
%TODO wstaw wzór anro-kin no 87
\end{frame}

\begin{frame}
{Drzewo transformacji Robota w lokalizacji}
	\begin{columns}
		\begin{column}{0.5\textwidth}
				W kontekście lokalizacji interesująca jest jedynie początkowa część drzewa transformacji opisująca połączenie układu świata z pierwszym układem związanym z robotem.
	Najczęściej między tymi dwoma jest wykonywane przejście przez układ początkowy mapy oraz układ rozpoczęcia odometrii, jeśli jest wykorzystywana.
		\end{column}
		\begin{column}{0.5\textwidth}  %%<--- here
			%TODO obrazek drzewa transformacji Velmy
		\end{column}
	\end{columns}
\end{frame}

\begin{frame}
{Odometria}
	\begin{columns}
		\begin{column}{0.5\textwidth}
			\begin{itemize}
				\item metoda lokalizacji zliczeniowej
				\item oszacowanie pozycji robota na podstawie oszacowanej prędkości, kierunku i czasu ruchu
				\item względnie proste zadanie obliczeniowo, ale generuje bez przerwy akumulowane błędy
			\end{itemize}
		\end{column}
		\begin{column}{0.5\textwidth}  %%<--- here
			%TODO obrazek Robotics Vision and Control 6.4/159
		\end{column}
	\end{columns}
\end{frame}

\begin{frame}
{lokalizacja ze znacznikami}
	\begin{itemize}
		\item wykorzystuje punkty o znanym położeniu względem początkowego układu współrzędnych mapy, które robot jest w stanie rozróżnić i zidentyfikować
		\item mierząc pozycję robota kolejno względem punktów można obliczyć jego pozycję względem początkowego układu mapy
		\item ta metoda również zwraca pozycję z pewnym błędem z powodu niepewności zapamiętajej pozycji znaczników jak i niepewności pomiarowych pozycji względem robota, lecz ten błąd jest ograniczony, im lepiej umieszczone znaczniki tym mniejszy
	\end{itemize}
\end{frame}

\begin{frame}
{Lokalizacja z systemem LiDAR}
	\begin{itemize}
		\item problem podobny do poprzedniego, lecz bez rozpoznawania znaczników
		\item wykorzystanie algorytmu ICP (iterative closed point) do dopasowania aktualnego skanu robota do stworzonej uprzednio mapy
		\item niezależność od specjalnych znaczników, lecz większy nakład obliczeniowy
		\item w przypadku gdy zbudowana mapa składa się z podobnych fragmentów może się okazać, że algorytm zawiedzie (przykładowo jazda długim korytarzem bez elementów szczególnych)
	\end{itemize}
\end{frame}



\section{Slam}
\begin{frame}
{Połączenie budowy mapy i lokalizacji}
	\begin{itemize}
		\item W trakcie budowy modelu środowiska często niezbędne jest poruszenie robotem, aby uzyskać wszystkie dane
		\item Poruszenie robota wymaga algorytmu lokalizacji
		\item algorytmy lokalizacji wymagają mapy do minimalizacji błędu uzyskanej pozycji
		\item W celu rozwiązania tego problemu powstał algorytm SLAM (Simulataneous Localization and Mapping)
		\item Zadanie jednoczesnej lokalizacji i budowy mapy trzeba rozwiązywać, gdy mapa środowiska jest niedostępna, a pozycja robota nie jest znana
	\end{itemize}
\end{frame}

\section{Planowanie ścieżki}
\begin{frame}
{Problem planowania ścieżki}
\begin{itemize}
	\item problem popularny również w innych dziedzinach, przykładowo w grach komputerowych \cite{robotics_and_games}
	\item może wykorzystywać standardowe algorytmy przeszukiwania grafu jak A*, Djikstra, zwykle implementowane w zadaniu planowania globalnego
	\item w systemie ROS gotowe do wykorzystania są biblioteki implementujące między innymi algorytm DWA (Dynamic Window Approach), EBand (elastic band) oraz TEB (Timed Elastic Band). Biblioteki te oprócz wyznaczania ścieżki generują pożądane prędkości bazy mobilnej do jej wykonania.
\end{itemize}
\end{frame}

\begin{frame}
{Analiza problemu planowania ścieżki}
	Pierwszym punktem jest wybór reprezentacji mapy.
	Szkieletyzacja generuje reprezentację topologiczną środowiska, natomiast dekompozycja komórkowa dzieli przestrzeń wolną od przeszkód na wolne komórki	
	\begin{columns}
		\begin{column}{0.5\textwidth}
			\begin{figure}
				\begin{center}
					\includegraphics[page={3},clip, trim=4cm 15cm 4cm 2cm, scale=0.25]{pdf/A-Comprehensive-Study.pdf}
					\hspace*{5pt}\hbox{\scriptsize{Źródło:\thinspace{\footnotesize{\itshape{Zeyad Abd Algfoor, ... \cite{robotics_and_games}}}}}}
					\caption{Przykłady dekompozycji komórkowej }
				\end{center}
			\end{figure}
		\end{column}
		\begin{column}{0.5\textwidth}  %%<--- here
						\begin{figure}
				\begin{center}
					\includegraphics[page={5},clip, trim=4cm 15cm 4cm 2cm, scale=0.25]{pdf/A-Comprehensive-Study.pdf}
					\hspace*{5pt}\hbox{\scriptsize{Źródło:\thinspace{\footnotesize{\itshape{Zeyad Abd Algfoor, ... \cite{robotics_and_games}}}}}}
					\caption{Przykłady topologii}
				\end{center}
			\end{figure}
		\end{column}
	\end{columns}
\end{frame}

\begin{frame}
{Algorytm generowania ścieżki}
	Po wybraniu odpowiedniej reprezentacji mapy należy zdecydować, który algorytm wyszukiwania drogi zostanie wykorzystany.
	Do algorytmów wspomnianych wcześniej można dodać sztuczne pola potencjału, algorytm genetyczny i wiele innych.
\end{frame}

\begin{frame}
{Istotne szczegóły w kontekście robotyki}
	\begin{itemize}
		\item ścieżka zwykle ma prowadzić od miejsca, w którym znajduje się początkowy układ współrzędnych bazy mobilnej do pozycji sprecyzowanej przez użytkownika
		\item może się okazać, że wyznaczona ścieżka jest zbyt blisko przeszkody i robot w nią uderzy
		\item w celu uniknięcia takich wypadków stosowana jest mapa kosztów i ślad robota (\textit{ang.} footprint)
	\end{itemize}
\end{frame}

\section{Navigation Stack}
\begin{frame}
{ROS Navigation Stack}
	Na następnym slajdzie przedstawione jest podejście do ogólnego problemu nawigacji zaimplementowane i wspierane przez system ROS. 
	Istotnym jest rozdział na część globalną i lokalną.
	Globalna ścieżka wykorzystuje szybkie algorytmy nie zwracające uwagi na przeszkody dynamiczne.
	Lokalny plan generowany na podstawie globalnego jest w stanie na nie reagować.
	Dodatkowo należy zaimplementować system odratowania pozwalający na wyrwanie robota z sytuacji w której planer nie jest w stanie znaleźć drogi do celu.
\end{frame}

\begin{frame}
{Reprezentacja Navigation Stack w systemie ROS}
%TODO obrazek Navigation Stack
\end{frame}

{
\setbeamertemplate{headline}{}
\addtobeamertemplate{frametitle}{\vspace*{-4\baselineskip}}{}
\begin{frame}
\frametitle{Stan pracy}
	\begin{center}
		\LARGE{\textbf{STAN PRACY}}
	\end{center}
\end{frame}
}

\section{Sformułowanie zadań}
\begin{frame}
{Zadania do wykonania}
Problem ogólny: implementacja mechanizmu nawigacji do symulacji mobilnej bazy wielokierunkowej robota Velma

Założenia:
	\begin{itemize}
		\item budowa mapy na płaskim podłożu i w zamkniętej przestrzeni
		\item mechanizm lokalizacji robota w stworzonym uprzednio środowisku oraz mechanizm jednoczesnej budowy mapy i lokalizacji
		\item mechanizm stworzenia i wykonania ścieżki między dowolnymi dwoma punktami w stworzonym środowisku
	\end{itemize}
\end{frame}

\begin{frame}
{Wykorzystywane narzędzia}
	\begin{columns}
		\begin{column}{0.25\textwidth}
			\begin{figure}
				\begin{center}
					\includegraphics[width=\textwidth]{img/Ros_logo.png}
				\end{center}
			\end{figure}
		\end{column}
		\begin{column}{0.25\textwidth}  %%<--- here
						\begin{figure}
				\begin{center}
					\includegraphics[width=\textwidth]{img/Gazebo_logo.png}		
				\end{center}
			\end{figure}
		\end{column}
				\begin{column}{0.25\textwidth}
			\begin{figure}
				\begin{center}
					\includegraphics[width=\textwidth]{img/Python_logo.png}			
				\end{center}
			\end{figure}
		\end{column}
		\begin{column}{0.25\textwidth}  %%<--- here
						\begin{figure}
				\begin{center}
					\includegraphics[width=\textwidth]{img/C++_Logo.png}
				\end{center}
			\end{figure}
		\end{column}
	\end{columns}
\footnotesize{Żródła: wikipedia.org, pngkey.com}
\end{frame}



\section{Dotychczasowe wyniki}
\begin{frame}
{Dotychczasowe wyniki pracy}
	\begin{itemize}
		\item naprawa drzewa transformacji
		\item zaimplementowanie budowy mapy 3D oraz 2D
		\item zaimplementowanie lokalizacji z pomocą czujników LiDAR
	\end{itemize}
\end{frame}

\section{Zadania do wykonania}
\begin{frame}
{Problem wykrywania przeszkód}
	\begin{itemize}
		\item problem wykrycia blatu stołu
		\item czujniki w bazie mobilnej są w stanie zarejestrować nogi stołu, lecz blatu już nie zauważą
		\item należy wykorzystać kamerę Kinect do wykrycia przeszkód poza zasięgiem czujników LiDAR
		\item istnieje opcja stworzenia oddzielnej mapy kosztów z danych czujników laserowych i oddzielnej z danych kamery Kinect lub połączenia danych w jedną mapę
	\end{itemize}
\end{frame}

\begin{frame}
{Problem algorytmu planowania lokalnego}
	\begin{itemize}
		\item obecnie algorytm planowania lokalnego traktuje bazę jako różnicową, a nie wielokierunkową (holonomiczną)
		\item zostaną przeprowadzone próby dostosowania obecnego planera do możliwości bazy jezdnej
		\item zostaną wyszukane i zaimplementowane inne algorytmy w celu sprawdzenia który jest lepszy
	\end{itemize}
\end{frame}

\begin{frame}
{Przeprowadzanie testów}
	\begin{itemize}
		\item testy z dostępnymi algorytmami lokalizacji, budowy mapy oraz ścieżki
		\item przy lokalizacji sprawdzany będzie błąd względem idealnej pozycji udostępnianiej przez symulator Gazebo
		\item przy planowaniu ścieżki sprawdzany będzie czas wykonania ściezki, reakcja na przeszkody dynamiczne, jakość systemu odratowania i wykorzystanie wszystkich stopni swobody bazy mobilnej
	\end{itemize}
\end{frame}

\section{Zakończenie}
\begin{frame}
	\begin{center}
	\LARGE{\textbf{DZIĘKUJĘ}}
	\end{center}
\end{frame}

%TODO dodaj bibliografię


\begin{frame}
\setbeamertemplate{headline}{}
\addtobeamertemplate{frametitle}{\vspace*{-2\baselineskip}}{}
\frametitle{Źródła}
\printbibliography
\end{frame}

%\resumenavigation
\end{document}