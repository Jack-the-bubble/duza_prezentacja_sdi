\begin{frame}
{Zadania do wykonania}
Problem ogólny: implementacja mechanizmu nawigacji do symulacji mobilnej bazy wielokierunkowej robota Velma

Założenia:
	\begin{itemize}
		\item budowa mapy na płaskim podłożu i w zamkniętej przestrzeni
		\item mechanizm lokalizacji robota w stworzonym uprzednio środowisku oraz mechanizm jednoczesnej budowy mapy i lokalizacji
		\item mechanizm stworzenia i wykonania ścieżki między dowolnymi dwoma punktami w stworzonym środowisku
	\end{itemize}
\end{frame}

\begin{frame}
{Wykorzystywane narzędzia}
	\begin{columns}
		\begin{column}{0.25\textwidth}
			\begin{figure}
				\begin{center}
					\includegraphics[width=\textwidth]{img/Ros_logo.png}
				\end{center}
			\end{figure}
		\end{column}
		\begin{column}{0.25\textwidth}  %%<--- here
						\begin{figure}
				\begin{center}
					\includegraphics[width=\textwidth]{img/Gazebo_logo.png}		
				\end{center}
			\end{figure}
		\end{column}
				\begin{column}{0.25\textwidth}
			\begin{figure}
				\begin{center}
					\includegraphics[width=\textwidth]{img/Python_logo.png}			
				\end{center}
			\end{figure}
		\end{column}
		\begin{column}{0.25\textwidth}  %%<--- here
						\begin{figure}
				\begin{center}
					\includegraphics[width=\textwidth]{img/C++_Logo.png}
				\end{center}
			\end{figure}
		\end{column}
	\end{columns}
\footnotesize{Żródła: wikipedia.org, pngkey.com}
\end{frame}

