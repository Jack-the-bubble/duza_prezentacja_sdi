\begin{frame}
{Problem wykrywania przeszkód}
	\begin{itemize}
		\item problem wykrycia blatu stołu
		\item czujniki w bazie mobilnej są w stanie zarejestrować nogi stołu, lecz blatu już nie zauważą
		\item należy wykorzystać kamerę Kinect do wykrycia przeszkód poza zasięgiem czujników LiDAR
	\end{itemize}
\end{frame}

\begin{frame}
{Problem algorytmu planowania lokalnego}
	\begin{itemize}
		\item obecnie udało się dostosować dwa z trzech docelowych algorytmów lokalnych do wykorzystania wielokierunkowości bazy, algorytm EBand wciąż traktuje bazę mobilną jako różnicową
		\item zostaną przeprowadzone próby dostosowania wspomnianego planera do możliwości bazy jezdnej
		\item zostaną przeprowadzone testy w celu sprawdzenia, który planer lokalny jest lepszy
	\end{itemize}
\end{frame}

\begin{frame}
{Przeprowadzanie testów}
	\begin{itemize}
		\item testy z dostępnymi algorytmami lokalizacji, budowy mapy oraz ścieżki
		\item przy lokalizacji sprawdzany będzie błąd względem idealnej pozycji udostępnianiej przez symulator Gazebo
		\item przy planowaniu ścieżki sprawdzany będzie czas wykonania ściezki, reakcja na przeszkody dynamiczne, jakość systemu odratowania i gładkość wykonanej ścieżki
	\end{itemize}
\end{frame}