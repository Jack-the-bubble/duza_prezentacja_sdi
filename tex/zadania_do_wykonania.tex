\begin{frame}
{Problem wykrywania przeszkód}
	\begin{itemize}
		\item problem wykrycia blatu stołu
		\item czujniki w bazie mobilnej są w stanie zarejestrować nogi stołu, lecz blatu już nie zauważą
		\item należy wykorzystać kamerę Kinect do wykrycia przeszkód poza zasięgiem czujników LiDAR
		\item istnieje opcja stworzenia oddzielnej mapy kosztów z danych czujników laserowych i oddzielnej z danych kamery Kinect lub połączenia danych w jedną mapę
	\end{itemize}
\end{frame}

\begin{frame}
{Problem algorytmu planowania lokalnego}
	\begin{itemize}
		\item obecnie algorytm planowania lokalnego traktuje bazę jako różnicową, a nie wielokierunkową (holonomiczną)
		\item zostaną przeprowadzone próby dostosowania obecnego planera do możliwości bazy jezdnej
		\item zostaną wyszukane i zaimplementowane inne algorytmy w celu sprawdzenia który jest lepszy
	\end{itemize}
\end{frame}

\begin{frame}
{Przeprowadzanie testów}
	\begin{itemize}
		\item testy z dostępnymi algorytmami lokalizacji, budowy mapy oraz ścieżki
		\item przy lokalizacji sprawdzany będzie błąd względem idealnej pozycji udostępnianiej przez symulator Gazebo
		\item przy planowaniu ścieżki sprawdzany będzie czas wykonania ściezki, reakcja na przeszkody dynamiczne, jakość systemu odratowania i wykorzystanie wszystkich stopni swobody bazy mobilnej
	\end{itemize}
\end{frame}