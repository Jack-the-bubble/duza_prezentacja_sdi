
\begin{frame}
{Operator przekształcenia jednorodnego}
Macierz przedstawiająca operację przejścia z układu współrzędnych $0$ do $1$.
Dzięki temu można przekształcić pozycję obiektu znajdującego się w układzie $1$ i przedstawić jako pozycję względem układu $0$.

%TODO wstawić r-nie anro-kin no 34
\end{frame}

\begin{frame}
{Drzewo transformacji}
	\begin{itemize}
		\item 4 parametry Denavita-Hartenberga opisujące przejście między dwoma układami współrzędnych (obrót i przesunięcie względem odpowiednich osi $x$ oraz $z$)
		\item macierze wygenerowane z tych parametrów można mnożyć sekwencyjnie, dzięki czemu relatywnie proste jest przedstawianie układów szeregowych z dużą ilością stawów - drzewo powstaje przez przejście z podstawowego układu (przykładowo świata) przez kolejne układy do docelowego
		\item drzewo transformacji może mieć tylko jeden korzeń, każdy węzeł (układ współrzędnych) tylko jednego rodzica, za to wiele dzieci
	\end{itemize}
	
%TODO wstaw wzór anro-kin no 87
\end{frame}

\begin{frame}
{Drzewo transformacji Robota w lokalizacji}
	\begin{columns}
		\begin{column}{0.5\textwidth}
				W kontekście lokalizacji interesująca jest jedynie początkowa część drzewa transformacji opisująca połączenie układu świata z pierwszym układem związanym z robotem.
	Najczęściej między tymi dwoma jest wykonywane przejście przez układ początkowy mapy oraz układ rozpoczęcia odometrii, jeśli jest wykorzystywana.
		\end{column}
		\begin{column}{0.5\textwidth}  %%<--- here
			%TODO obrazek drzewa transformacji Velmy
		\end{column}
	\end{columns}
\end{frame}

\begin{frame}
{Odometria}
	\begin{columns}
		\begin{column}{0.5\textwidth}
			\begin{itemize}
				\item metoda lokalizacji zliczeniowej
				\item oszacowanie pozycji robota na podstawie oszacowanej prędkości, kierunku i czasu ruchu
				\item względnie proste zadanie obliczeniowo, ale generuje bez przerwy akumulowane błędy
			\end{itemize}
		\end{column}
		\begin{column}{0.5\textwidth}  %%<--- here
			%TODO obrazek Robotics Vision and Control 6.4/159
		\end{column}
	\end{columns}
\end{frame}

\begin{frame}
{lokalizacja ze znacznikami}
	\begin{itemize}
		\item wykorzystuje punkty o znanym położeniu względem początkowego układu współrzędnych mapy, które robot jest w stanie rozróżnić i zidentyfikować
		\item mierząc pozycję robota kolejno względem punktów można obliczyć jego pozycję względem początkowego układu mapy
		\item ta metoda również zwraca pozycję z pewnym błędem z powodu niepewności zapamiętajej pozycji znaczników jak i niepewności pomiarowych pozycji względem robota, lecz ten błąd jest ograniczony, im lepiej umieszczone znaczniki tym mniejszy
	\end{itemize}
\end{frame}

\begin{frame}
{Lokalizacja z systemem LiDAR}
	\begin{itemize}
		\item problem podobny do poprzedniego, lecz bez rozpoznawania znaczników
		\item wykorzystanie algorytmu ICP (iterative closed point) do dopasowania aktualnego skanu robota do stworzonej uprzednio mapy
		\item niezależność od specjalnych znaczników, lecz większy nakład obliczeniowy
		\item w przypadku gdy zbudowana mapa składa się z podobnych fragmentów może się okazać, że algorytm zawiedzie (przykładowo jazda długim korytarzem bez elementów szczególnych)
	\end{itemize}
\end{frame}

