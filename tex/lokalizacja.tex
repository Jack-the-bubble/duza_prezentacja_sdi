
\begin{frame}
{Macierz przekształcenia jednorodnego}
Poniżej znajduje się jedna macierz przekształcenia jednorodnego. 
Można ją interpretować na wiele sposobów, jednym z nich jest przedstawienie operacji przejścia z układu współrzędnych $0$ do $1$. 
Dzięki temu można przekształcić pozycję obiektu znajdującego się w układzie $1$ i przedstawić jako pozycję względem układu $0$.
Kolejnym przykładem jest możliwość odczytania położenia ($\prescript{0}{1}P_{3x1}$) i orientacji ($\prescript{0}{1}R_{3x3}$) układu $1$ w układzie $0$.
\begin{equation*}
	\prescript{0}{1}T = 
	\begin{bmatrix}
		r_{}11 	& r_{}12	&	r_{}13	& \prescript{0}{1}P_x \\
		r_{}21 	& r_{}22	&	r_{}23	& \prescript{0}{1}P_y \\
		r_{}31 	& r_{}32	& r_{}33	& \prescript{0}{1}P_z \\
		0				&	0				&	0				&	1										\\
	\end{bmatrix}
	\Rightarrow
		\prescript{0}{1}T = 
	 \left[ \begin{array}{c|c}
   \prescript{0}{1}R_{3x3} 	& \prescript{0}{1}P_{3x1} \\
   \midrule
   0_{1x3}									& 1												\\
\end{array}\right]
\end{equation*}
\end{frame}



\begin{frame}
{Drzewo transformacji}
	\begin{itemize}
		\item 
		Posiadając macierze $\prescript{0}{1}T$, $\prescript{1}{2}T$, $\prescript{2}{3}T$, $\prescript{3}{4}T$, a więc proste drzewo transformacji, można uzyskać pozycję punktu, który początkowo przedstawiony jest w układzie współrzędnych $4$, w układzie $0$ poprzez kolejne przemnożenia przedstawionych macierzy. 
		\begin{equation*}
			\prescript{0}{}P = \prescript{0}{1}T \cdot \prescript{1}{2}T \cdot \prescript{2}{3}T \cdot \prescript{3}{4}T \cdot \prescript{4}{}P
		\end{equation*}
%		macierze opisujące przejście między kolejnymi układami odniesienia można mnożyć sekwencyjnie, dzięki czemu relatywnie proste jest przedstawianie układów szeregowych z dużą ilością stawów - drzewo powstaje przez przejście z podstawowego układu (przykładowo świata) przez kolejne układy do docelowego
		\item drzewo transformacji może mieć tylko jeden korzeń, każdy węzeł (układ współrzędnych) tylko jednego rodzica, za to wiele dzieci
	\end{itemize}
	
	
%	\begin{equation*}
%		\prescript{i-1}{i}T = 
%		\begin{bmatrix}
%			c\theta_i	&	-s\theta_i	& 0 &	a_{i-1}	\\
%			s\theta_ic\alpha_{i-1}	&	c\theta_ic\alpha_{i-1}	&	-s\alpha_{i-1}	&	-d_is\alpha_{i-1}	\\
%			s\theta_is\alpha_{i-1}	&	c\theta_is\alpha_{i-1}	&	c\alpha_{i-1}		&	d_ic\alpha_{i-1}\\
%			0	&	0	&	0	&	1	\\
%		\end{bmatrix}
%	\end{equation*}
\end{frame}

\begin{frame}
{Drzewo transformacji robota w zadaniu lokalizacji}
	\begin{columns}
		\begin{column}{0.5\textwidth}
				W kontekście lokalizacji interesująca jest jedynie początkowa część drzewa transformacji opisująca połączenie układu świata z pierwszym układem związanym z robotem.
	Najczęściej między tymi dwoma jest wykonywane przejście przez układ początkowy mapy oraz układ odniesienia lokalizacji, jeśli zaimplementowano ją metodą odometrii.
		\end{column}
		\begin{column}{0.5\textwidth}  %%<--- here
			\begin{figure}
				\begin{center}
					\includegraphics[height=0.6\textheight]{img/velma_tf.png} 
					\caption{początek drzewa transformacji po uruchomieniu symulacji robota}
				\end{center}
			\end{figure}
		\end{column}
	\end{columns}
\end{frame}

\begin{frame}
{Odometria}
	\begin{columns}
		\begin{column}{0.45\textwidth}
			\begin{itemize}
				\item metoda lokalizacji zliczeniowej
				\item obliczenie względnej pozycji robota na podstawie oszacowanej prędkości, kierunku i czasu ruchu
				\item względnie proste zadanie obliczeniowo (w porównaniu do lokalizacji z pomocą czujników LiDAR), ale generuje systematycznie akumulowane błędy
			\end{itemize}
		\end{column}
		\begin{column}{0.55\textwidth}  %%<--- here
			\begin{figure}
				\begin{center}
					\includegraphics[page={180},clip, trim=8cm 1.5cm 3cm 18cm, scale=0.6]{pdf/2017_Book_RoboticsVisionAndControl.pdf}
					\hspace*{15pt}\hbox{\scriptsize{Źródło:\thinspace{\footnotesize{\itshape{Robotics Vision and Control \cite{robotics_vision}}}}}}
					\caption{Błędy odometrii. Niebieska droga oznacza rzeczywistą ścieżkę, czerwona ścieżka obliczoną }
				\end{center}
			\end{figure}
		\end{column}
	\end{columns}
\end{frame}

\begin{frame}
{Lokalizacja z wykorzystaniem znaczników}
	\begin{itemize}
		\item wykorzystuje punkty o znanym położeniu względem początkowego układu współrzędnych mapy, które robot jest w stanie rozróżnić i zidentyfikować
		\item mierząc pozycję robota względem punktów można obliczyć jego pozycję względem początkowego układu mapy
		\item ta metoda również zwraca pozycję z pewnym błędem z powodu niepewności zmierzonej pozycji znaczników, jak i niepewności pomiarowych pozycji ich względem robota, lecz ten błąd jest ograniczony, im lepiej umieszczone znaczniki, tym mniejszy
	\end{itemize}
\end{frame}

\begin{frame}
{Lokalizacja z wykorzystaniem systemu LiDAR}
	\begin{itemize}
		\item problem podobny do poprzedniego, lecz bez rozpoznawania znaczników
		\item wykorzystanie algorytmu ICP (iterative closed point) do dopasowania aktualnego skanu robota do stworzonej uprzednio mapy
		\item niezależność od specjalnych znaczników, lecz większy nakład obliczeniowy
		\item w przypadku gdy zbudowana mapa składa się z podobnych fragmentów może się okazać, że algorytm zawiedzie (przykładowo jazda długim korytarzem bez elementów szczególnych)
	\end{itemize}
\end{frame}

