\begin{frame}
{ROS Navigation Stack}
	\begin{itemize}
		\item Na następnej stronie przedstawione jest podejście do ogólnego problemu nawigacji zaimplementowane i wspierane przez system ROS
		\item Istotny jest podział na część globalną i lokalną
		\item Globalna ścieżka wykorzystuje szybkie algorytmy, które nie sprawdzają stanu przeszkód dynamicznych
		\item Lokalny plan jest generowany na podstawie globalnego, wykorzystuje bardziej złożone algorytmy reagujące na przeszkody dynamiczne, które biorą pod uwagę jedynie otoczenie w pobliżu robota
		\item Dodatkowo należy zaimplementować system odratowania pozwalający na wyrwanie robota z sytuacji, w której algorytm lokalny nie jest w stanie wyznaczyć ścieżki lub wyjść z oscylacji
	\end{itemize}
\end{frame}

\begin{frame}
{Reprezentacja Navigation Stack w systemie ROS}
	\begin{center}
		\begin{figure}
			\centering
			\includegraphics[page={29},clip, trim=0cm 0.5cm 0cm 2cm, scale=0.7]{pdf/Wprowadzenie-do-bloku-robotow-mobilnych.pdf}
			\hspace*{15pt}\hbox{\scriptsize{Źródło:\thinspace{\footnotesize{\itshape{\href{http://wiki.ros.org/move_base}{ros.org/move\_base}			
			}}}}}
		\end{figure}
	\end{center}
\end{frame}