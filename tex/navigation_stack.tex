\begin{frame}
{ROS Navigation Stack}
	Na następnym slajdzie przedstawione jest podejście do ogólnego problemu nawigacji zaimplementowane i wspierane przez system ROS. 
	Istotnym jest rozdział na część globalną i lokalną.
	Globalna ścieżka wykorzystuje szybkie algorytmy nie zwracające uwagi na przeszkody dynamiczne.
	Lokalny plan generowany na podstawie globalnego jest w stanie na nie reagować.
	Dodatkowo należy zaimplementować system odratowania pozwalający na wyrwanie robota z sytuacji w której planer nie jest w stanie znaleźć drogi do celu.
\end{frame}

\begin{frame}
{Reprezentacja Navigation Stack w systemie ROS}
	\begin{center}
		\begin{figure}
			\centering
			\includegraphics[page={29},clip, trim=0cm 0.5cm 0cm 2cm, scale=0.7]{pdf/Wprowadzenie-do-bloku-robotow-mobilnych.pdf}
			\hspace*{15pt}\hbox{\scriptsize{Źródło:\thinspace{\footnotesize{\itshape{Wojciech Dudek}}}}}
		\end{figure}
	\end{center}
\end{frame}