\begin{frame}
{Zastosowania wyszukiwania ścieżki}
\begin{itemize}
	\item problem popularny również w innych dziedzinach, przykładowo w grach komputerowych
%TODO A Comprehensive Study on Pathfinding Techniques for Robotics and Video Games
	\item może wykorzystywać standardowe algorytmy jak A*, Djikstra
\end{itemize}
\end{frame}

\begin{frame}
{Analiza problemu wyszukiwania ścieżki}
	Pierwszym punktem jest wybór reprezentacji mapy.
	Szkieletyzacja generuje reprezentację topologiczną środowiska, natomiast dekompozycja komórkowa dzieli przestrzeń wolną od przeszkód na wolne komórki	
%TODO obrazek reprezentacji grafowej i dekompozycji
\end{frame}

\begin{frame}
{Algorytm generowania ścieżki}
	Po wybraniu odpowiedniej reprezentacji należy zdecydować który algorytm wyszukiwania drogi zostanie wykorzystany.
	Do algorytmów wspomnianych wcześniej można dodać sztuczne pola potencjału, algorytm genetyczny i wiele innych.
\end{frame}

\begin{frame}
{Istotne szczegóły w kontekście robotyki}
	\begin{itemize}
		\item ścieżka zwykle ma prowadzić od miejsca w którym znajduje się początkowy układ współrzędnych bazy mobilnej do pozycji sprecyzowanej przez użytkownika
		\item może się okazać, że wyznaczona ścieżka jest zbyt blisko przeszkody i robot w nią uderzy
		\item w celu uniknięcia takich wypadków stosowana jest mapa kosztów i ślad robota (\textit{ang.} footprint)
	\end{itemize}
\end{frame}