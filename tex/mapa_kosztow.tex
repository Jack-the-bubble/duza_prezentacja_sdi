
\begin{frame}{Czym jest mapa kosztów}
	\begin{itemize}
		\item mapa - informacje które punkty w przestrzeni są zajęte, ewentualnie informacje o ich identyfikacji
		\item mapa kosztów - każdy punkt w przestrzeni ma przypisaną pewną wartość (binarnie lub z większym przedziałem), wykorzystywana w algorytmach wyszukiwania ścieżki
		\item przez dodawanie wartości pośrednich można sprawić by algorytm unikał nieporządanych miejsc jak pobliże ścian czy śliska nawierzchnia
	\end{itemize}
\end{frame}

\begin{frame}{Mapa i mapa kosztów}
\begin{columns}
		\begin{column}{0.5\textwidth}
			\begin{center}
				MAPA
			\end{center}
%TODO			wstaw obrazek mapy
		\end{column}
		\begin{column}{0.5\textwidth}  %%<--- here
			\begin{center}
				MAPA KOSZTÓW
			\end{center}
%TODO			wstaw obrazek mapy kosztów
		\end{column}
	\end{columns}
\end{frame}

\begin{frame}
{Wielowarstwowe mapy kosztów}
\begin{columns}
		\begin{column}{0.5\textwidth}
			\begin{itemize}
				\item pozwalają na zobrazowanie kosztu poruszania w większej ilości kontekstów (ruch prawostronny, omijanie przestrzeni publicznej)
				\item możliwość wykrywania przeszkód na kilku poziomach (przykładowo wykrycie blatu stołu)
				\item możliwość wykorzystania chmury punktów z Kinecta do stworzenia mapy kosztów przez zrzutowanie wszystkich punktów na płaszczyznę
			\end{itemize}
		\end{column}
		\begin{column}{0.5\textwidth}  %%<--- here
%TODO			wstaw obrazek mapy wielowarstwowej
		\end{column}
	\end{columns}
\end{frame}

